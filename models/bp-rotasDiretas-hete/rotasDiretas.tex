\documentclass[12pt]{article}

\usepackage{amsmath}
\usepackage{amsfonts}
\usepackage[brazil]{babel}
\usepackage[latin1]{inputenc}
\usepackage[T1]{fontenc}
\usepackage{geometry}
\geometry{letterpaper,left=2.5cm,right=2.5cm,top=2.0cm,bottom=1.0cm}
\usepackage{graphicx}
\sloppy

\begin{document}

\begin{center}
\textbf{Modelo de Rotas Diretas}
\end{center}

\vspace{5mm}

\noindent Sejam os conjuntos de rotas que t\^em o Cross-Docking como ponto de partida e retorno e $k \in K$ um ve\'iculo da frota:
\begin{itemize}
  \item $R_f^k$: Conjunto de todas as rotas para visitar os v\'ertices fornecedores com o ve\'iculo $k$
  \item $R_c^k$: Conjunto de todas as rotas para visitar os v\'ertices consumidores com o ve\'iculo $k$
  \item $R_d^k$: Conjunto de todas as rotas diretas com o ve\'iculo $k$ (visita um conjunto n\~ao vazio de fornecedores e logo ap\'os um conjunto n\~ao vazio de consumidores)
\end{itemize}


\noindent E as seguintes vari\'aveis de decis\~ao:
\begin{table}[!htb]
\begin{tabular}{rl}
\vspace{1mm}
$\lambda^k_r = $ &
$\left\{
\begin{array} {l}
1 - \textrm{ A rota } r \in R_f \textrm{ sera associada ao veculo } k \textrm{ no roteamento dos \underline{fornecedores}} \\
0 - \textrm{ Caso contrario}
\end{array} \right.
$
\\
\vspace{2mm}
$\gamma^k_r = $ &
$\left\{
\begin{array} {l}
1 - \textrm{ A rota } r \in R_c \textrm{ sera associada ao veculo } k \textrm{ no roteamento dos \underline{consumidores}} \\
0 - \textrm{ Caso contrario}
\end{array} \right.
$
\\
\vspace{2mm}
$\delta^k_r = $ &
$\left\{
\begin{array} {l}
1 - \textrm{ A rota } r \in R_d \textrm{ sera associada ao veculo } k \textrm{ no roteamento direto} \\
0 - \textrm{ Caso contrario}
\end{array} \right.
$
\\
\vspace{1mm}
$\tau^k_i = $ &
$\left\{
\begin{array} {l}
1 - \textrm{ O veiculo } k \textrm{ carrega/descarrega a mercadoria } i \textrm{ no Cross-Docking}\\
0 - \textrm{ Caso contrario } 
\end{array} \right.
$
\end{tabular}
\end{table}

\noindent Considere tamb\'em o par\^ametro $a^i_r$ que assume valor $1$ se o vertice $i$ \'e visitado pela rota $r$ e $0$ caso contr\'ario. O modelo de gera\c{c}\~ao de colunas para o VRPCD com rotas diretas \'e apresentado a seguir.

\[
Min \,\, \sum\limits_{k \in K}{ \sum\limits_{r \in R_f^k} {c_r^k \lambda^k_r}} + \sum\limits_{k \in K}{ \sum\limits_{r \in R_c^k} {c_r^k \gamma^k_r}} + \sum\limits_{k \in K}{ \sum\limits_{r \in R_d^k} {c_r^k \delta^k_r}} + \sum\limits_{k \in K} {\sum\limits_{p_i \in P} {c^k_i \tau^k_i}}
\]

\begin{eqnarray}
\label{eqA01}
  \sum\limits_{r \in R_f^k} {\lambda^k_r} + \sum\limits_{r \in R_d^k} {\delta^k_r} = 1  & \hspace{1cm} & \forall k \in K \\
\label{eqA02}
  \sum\limits_{r \in R_c^k} {\gamma^k_r} + \sum\limits_{r \in R_d^k} {\delta^k_r} = 1 & \hspace{1cm} & \forall k \in K \\
\label{eqA03}
  \sum\limits_{k \in K}{ \sum\limits_{r \in R_f^k} {a^i_r \lambda^k_r}} + \sum\limits_{k \in K}{ \sum\limits_{r \in R_d^k} {a^i_r \delta^k_r}} = 1 & \hspace{1cm} & \forall i \in F \\
\label{eqA04}
  \sum\limits_{k \in K}{ \sum\limits_{r \in R_c^k} {a^{i'}_r \gamma^k_r}} + \sum\limits_{k \in K}{ \sum\limits_{r \in R_d^k} {a^{i'}_r \delta^k_r}} = 1 & \hspace{1cm} & \forall i' \in C \\
\label{eqA05}
  \sum\limits_{r \in R_f^k} {\lambda^k_r a^i_r} - \sum\limits_{r \in R_c^k} {\gamma^k_r a^{i'}_r} + \tau^k_i \ge 0 & \hspace{1cm} & \forall p_i \in P, \forall k \in K \\
\label{eqA06}
  -\sum\limits_{r \in R_f^k} {\lambda^k_r a^i_r} + \sum\limits_{r \in R_c^k} {\gamma^k_r a^{i'}_r} + \tau^k_i \ge 0 & \hspace{1cm} & \forall p_i \in P, \forall k \in K \\
\label{eqA07}
  1 - \sum\limits_{r \in R_d^k} {\delta^k_r a^i_r} - \tau^k_i \ge 0 & \hspace{1cm} & \forall p_i \in P, \forall k \in K \\
\label{eqA07}
  \{\lambda^{|R_f||K|}, \gamma^{|R_c||K|}, \delta^{|R_d||K|}, \tau^{|K||P|}\} \in \mathbb{B}
\end{eqnarray}


\noindent Variaveis duais:\\

\begin{tabular}{llll}
$(1) \rightarrow \alpha \in \mathbb{R}^{|K|}$ & $(2) \rightarrow \beta \in \mathbb{R}^{|K|}$ & $(3) \rightarrow \theta \in \mathbb{R}^{|Q|}$ & $(4) \rightarrow \mu \in \mathbb{R}^{|Q|}$ \\
 & & & \\
$(5) \rightarrow \pi \in \mathbb{R}^{|K|\times|Q|}_+$ & $(6) \rightarrow \chi \in \mathbb{R}^{|K|\times|Q|}_+$ & $(7) \rightarrow \psi \in \mathbb{R}^{|K|\times|Q|}_+$ \\
\end{tabular}

\vspace{2cm}

\noindent Formulacao Dual: \vspace{5mm}

\[
Max \,\, \sum\limits_{k \in K}{\alpha^k} + \sum\limits_{k \in K}{\beta^k} + \sum\limits_{q \in Q}{\theta_q} + \sum\limits_{q \in Q}{\mu_q}
\]

\begin{small}
\begin{eqnarray}
\label{eqC01}
  \alpha^k + \sum\limits_{i \in F}{a^i_r \theta_i} + \sum\limits_{p_i \in P}{a^i_r \pi^k_q} - \sum\limits_{p_i \in P}{a^i_r \chi^k_q} \le c^k_r & \hspace{5mm} & \forall k \in K, \forall r \in R^k_f \\
\label{eqC02}
  \beta^k + \sum\limits_{i \in C}{a^i_r \mu_i} - \sum\limits_{p_i \in P}{a^i_r \pi^k_q} + \sum\limits_{p_i \in P}{a^i_r \chi^k_q} \le c^k_r & \hspace{5mm} & \forall k \in K, \forall r \in R^k_c \\
\label{eqC03}  
  \alpha^k + \beta^k + \sum\limits_{i \in F}{a^i_r \theta_i} + \sum\limits_{i \in C}{a^i_r \mu_i} - \sum\limits_{p_i \in P}{a^i_r \psi^k_i} \le c^k_r  & \hspace{5mm} & \forall k \in K, \forall r \in R^k_d \\
\label{eqC04}
  \pi^k_i + \chi^k_i - \psi^k_i \le c^k_i & \hspace{5mm} & \forall k \in K, \forall p_i \in P \\
\label{eqC05}
  \alpha^k, \beta^k, \theta_i, \mu_i \in \mathbb{R}  \\
\label{eqC06}
  \pi^k_i, \chi^k_i, \psi^k_i \in \mathbb{R}_+
\end{eqnarray}
\end{small}

\end{document}
