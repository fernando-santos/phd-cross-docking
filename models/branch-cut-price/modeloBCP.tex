\documentclass[12pt]{article}

\usepackage{amsmath}
\usepackage{amsfonts}
\usepackage[brazil]{babel}
\usepackage[latin1]{inputenc}
\usepackage[T1]{fontenc}
\usepackage{geometry}
\geometry{letterpaper,left=2.5cm,right=2.5cm,top=2.0cm,bottom=1.0cm}
\usepackage{graphicx}
\sloppy

\begin{document}

\begin{table}[!htb]
\begin{tabular}{rl}
\vspace{1mm}
$\lambda_r = $ &
$\left\{
\begin{array} {l}
1 - \textrm{ A rota } r \textrm{ sera percorrida por algum veiculo na solucao } \\
0 - \textrm{ Caso contrario}
\end{array} \right.
$
\\
\vspace{2mm}
$\gamma_{r'} = $ &
$\left\{
\begin{array} {l}
1 - \textrm{ A rota } r' \textrm{ sera percorrida por algum veiculo na solucao } \\
0 - \textrm{ Caso contrario}
\end{array} \right.
$
\\
\vspace{1mm}
$\tau_{rr'} = $ &
$\left\{
\begin{array} {l}
1 - \textrm{ O par de rotas } \{r,r'\} \textrm{ sera percorrido pelo mesmo veiculo na solucao }\\
0 - \textrm{ Caso contrario } 
\end{array} \right.
$
\\
\vspace{1mm}
$\tau_i = $ &
$\left\{
\begin{array} {l}
1 - \textrm{ A mercadoria } p_i \textrm{ foi coletada e entregue pelo mesmo veiculo }\\
0 - \textrm{ Caso contrario } 
\end{array} \right.
$
\end{tabular}
\end{table}

\noindent Formulacao linear inteira para o VRPCD:

\[
Min \,\, \sum\limits_{r \in R} {c_r \lambda_r} + \sum\limits_{r' \in R'} {c_{r'} \gamma_{r'}} + \sum\limits_{p_i \in P}{ c_i \tau_i }
\]

\begin{eqnarray}
\label{eqA01}
  \sum\limits_{r \in R} {\lambda_r} = K  \\
\label{eqA02}
  \sum\limits_{r' \in R'} {\gamma_{r'}} = K  \\
\label{eqA03}
  \sum\limits_{r \in R} {a_{ir} \lambda_r} = 1 & \hspace{1cm} & \forall i \in F \\
\label{eqA04}
  \sum\limits_{r' \in R'} {a_{ir'} \gamma_{r'}} = 1 & \hspace{1cm} & \forall i \in C \\
\label{eqA05}
  \lambda_r = \sum\limits_{r' \in R'}{\tau_{rr'}} & \hspace{1cm} & \forall r \in R \\
\label{eqA06}
  \gamma_{r'} = \sum\limits_{r \in R}{\tau_{rr'}} & \hspace{1cm} & \forall r' \in R' \\
\label{eqA07}
  \sum\limits_{r \in R}{\sum\limits_{r' \in R'}{a_{ir}a_{ir'}\tau_{rr'} \geq \tau_i}} & \hspace{1cm} & \forall p_i \in P \\
\label{eqA08}
  \sum\limits_{r \in R}{\sum\limits_{r' \in R'}{\tau_{rr'}}} = K  \\
\label{eqA09}
  \lambda_r, \gamma_{r'}, \tau_{rr'} \in \{0,1\} 
\end{eqnarray}

\vspace{2.5cm}
Ao relaxar a integralidade das variaveis de decisao e considerar $S$ e $S'$ subconjuntos respectivamente de $R$ e $R'$, obtem-se o RLMP. Associa-se entao variaveis duais as restricoes RLMP, para obter a formulacao dual. Variaveis duais:
\begin{table}[!htb]
\begin{center}
\begin{tabular}{lll}
$(1) \rightarrow \alpha \in \mathbb{R}$ & $(2) \rightarrow \beta \in \mathbb{R}$ & $(3) \rightarrow \theta \in \mathbb{R}^{|F|}$ \\
 & & \\
$(4) \rightarrow \mu \in \mathbb{R}^{|C|}$ & $(5) \rightarrow \psi \in \mathbb{R}^{|S|}$ & $(6) \rightarrow \phi \in \mathbb{R}^{|S'|}$ \\
 & & \\
$(7) \rightarrow \pi \in \mathbb{R}^{|P|}_+$ & $(8) \rightarrow \Delta \in \mathbb{R}$
\end{tabular}
\end{center}
\end{table}

\newpage
\noindent Formulacao dual: \vspace{5mm}

\[
\max \,\, K \alpha + K \beta + \sum\limits_{i \in F}{\theta_i} + \sum\limits_{i \in C}{\mu_i} + K \Delta
\]

\begin{small}
\begin{eqnarray}
\label{eqC01}
  \alpha + \sum\limits_{i \in F}{a_{ir} \theta_i} + \sum\limits_{r \in S}{\psi_r} \le c_r & \hspace{1cm} & \forall r \in S \\
\label{eqC02}
  \beta + \sum\limits_{i \in C}{a_{ir'} \mu_i} + \sum\limits_{r' \in S'}{\phi_{r'}} \le c_{r'} & \hspace{1cm} & \forall r' \in S' \\
\label{eqC03}
  -\psi_r - \phi_{r'} + \Delta \le t_{rr'} & \hspace{5mm} & \forall r \in S, \forall r' \in S'
\end{eqnarray}
\end{small}

\end{document}
