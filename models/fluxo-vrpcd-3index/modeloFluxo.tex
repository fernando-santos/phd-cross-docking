\documentclass[12pt]{article}
\usepackage{amsfonts}
\usepackage[brazil]{babel}
%\usepackage[latin1]{inputenc}
\usepackage[T1]{fontenc}
\usepackage{geometry}
\geometry{letterpaper,left=2.5cm,right=2.5cm,top=2.5cm,bottom=1.5cm}
\usepackage{graphicx}
\sloppy

\begin{document}

\begin{center}
\textbf{\textit{Commodities} devem passar pelo \emph{Cross-docking}} \\
\textbf{Modelo de Fluxo em Redes - 2} \\
\end{center}
\vspace{0.4cm}
\noindent Considere uma rede de distribuicao composta por um conjunto de fornecedores $F=\{1,2,..,N\}$, um de consumidores $C=\{1',2',..,N'\}$, onde $|F|=|C|$, e um deposito (cross-docking). E definido tambem um conjunto $E$ de encomendas a serem distribuidas na rede, definido por uma funcao bijetiva $E:F \times C$ que associa cada elemento $f \in F$ a um unico $c \in C$. Cada encomenda $(f,c) \in E$ esta associada a um peso $P_{(f,c)}$ e um custo para sua carga/descarga no cross-docking $c_{(f,c)}$.

\vspace{0.2cm}
Para a distribuicao das mercadorias tem-se disponivel um conjunto de veiculos $K$ todos com capacidade $P$. O problema de otimizacao consiste em definir o esquema de alocacao e roteamento dos veiculos para realizar com o minimo custo a distribuicao das mercadorias, sendo que todas devem ser coletadas nos fornecedores e passar pelo Cross-docking, antes de serem entregues aos respectivos consumidores. No Cross-docking nao e possivel armazenar mercadorias, apenas troca-las de veiculos.

\vspace{0.2cm}
Utiliza-se um grafo $G=(V,A)$, onde $V = F \cup C \cup \{0,0'\}$, sendo $0$ o deposito de cross-docking e $0'$ uma copia artificial deste deposito. O conjunto $A=\{(i,j), \, i,j \in V\}$ e o custo para atravessar uma aresta $(i,j) \in A$ \'e dado por $c_{ij}$. Para a formula\c{c}\~ao de fluxos sao definidos um conjunto de commodities $Q$ e uma funcao bijetora $g: Q \rightarrow \{F \cup C \}$ que associa a cada vertice uma unica commodity. O deposito oferta 2 unidades de cada commodity $q \in Q$, sendo que tanto o vertice $g(q) \in \{F \cup C \}$ quanto o deposito artificial apresentam demanda de 1 unidade de $q$. O fluxo das commodities pode seguir por diferentes rotas para atender as demandas, sendo que o total de rotas na solucao do problema deve ser $K$. As variaveis de decisao sao dadas por:

\vspace{0.8cm}
\begin{table}[!htb]
\begin{tabular}{rl}
$f^{qk}_{ij} = $ & Fluxo da commodity $q$ no arco $(i,j)$ associada a rota $k$\\
$x^k_{ij} = $ &
$\left\{
\begin{array} {l}
1 - \textrm{ A rota } k \textrm{ esta associada ao arco } (i,j) \textrm{, utilizado para o fluxo das commodities}\\
1 - \textrm{ O arco } (i,j) \textrm{ e ativado na rota } k \textrm{ para o fluxo das commodities}\\
0 - \textrm{ Caso contrario}
\end{array} \right.
$
\\
$y^{qk} = $ &
$\left\{
\begin{array} {l}
1 - \textrm{ A rota } k \textrm{ esta associada ao fluxo da commodity } q \\
0 - \textrm{ Caso contrario}
\end{array} \right.
$
\\
$u^k_{(f,c)} = $ &
$\left\{
\begin{array} {l}
1 -\textrm{ As demandas de } f \in F\textrm{ e } c \in C \, | \, (f,c) \in E \textrm{ sao atendidas pela mesma rota } k \\
0 - \textrm{ Caso contrario}
\end{array} \right.
$
\end{tabular}
\end{table}

\newpage
\begin{center}
 $Min \,\, \sum\limits_{k=1}^K {\sum\limits_{(i,j) \in A} {{c_{ij} x^k_{ij}}}} + \sum\limits_{(f,c) \in E} {c_{(f,c)} (1 - \sum\limits_{k=1}^K {u^k_{(f,c)})}}$
\end{center}

\begin{eqnarray}
\label{eq01}
  \sum\limits_{f \in F} {f^{qk}_{0f}} = 2 y^{qk} & \hspace{1cm} & \forall q \in Q, \forall k = \{1,..,K\} \\
\label{eq02}
  \sum\limits_{c \in C} {f^{qk}_{0c}} = 2 y^{qk} & \hspace{1cm} & \forall q \in Q, \forall k = \{1,..,K\} \\
\label{eq03}
  \sum\limits_{j \in \Gamma^+_i} {f^q_{ij}} - \sum\limits_{h \in \Gamma^-_i}{f^q_{hi}} = \left\{
\begin{array} {l}
-y^{qk} \textrm{, se } i = g(q)\\
0 \,\,\,\, \,\,\,\,\,\textrm{, c. c.}
\end{array} \right.
& \hspace{1cm} & \forall i \in \{F \cup C\}, \forall q \in Q, \forall k\\
\label{eq04}
  \sum\limits_{k=1}^K {y^{qk}} = 1 & \hspace{1cm} & \forall q \in Q \\
\label{eq05}
  \sum\limits_{f \in F} {x^k_{0f}} = 1 & \hspace{1cm} & \forall k = \{1,..,K\} \\
\label{eq06}
  \sum\limits_{c \in C} {x^k_{0c}} = 1 & \hspace{1cm} & \forall k = \{1,..,K\} \\
\label{eq07}
  \sum\limits_{q \in Q} {f^{qk}_{ij}} \le |V| x^k_{ij} & \hspace{1cm} & \forall (i,j) \in A, \forall k = \{1,..,K\} \\
\label{eq08}
  \sum\limits_{k=1}^K {\sum\limits_{j \in \Gamma^+_i}{x^k_{ij}}} + \sum\limits_{k=1}^K {\sum\limits_{h \in \Gamma^-_i}{x_{hi}}} = 2 & \hspace{1cm} & \forall i \in V - \{0,0'\} \\
\label{eq09}
  \sum\limits_{q \in Q} {y^{qk} P_{(g(q),c)}} \le P & \hspace{1cm} & \forall k = \{1,..,K\} \\
\label{eq10}
  \sum\limits_{q \in Q} {y^{qk} P_{(f,g(q))}} \le P & \hspace{1cm} & \forall k = \{1,..,K\} \\
\label{eq11}
  y^{fk} + y^{ck} \le u_{(f,c)} + 1 & \hspace{1cm} & \forall k = \{1,..,K\}, \forall (f,c) \in E \\
\label{eq12}
  y^{fk} + y^{ck} \ge 2u_{(f,c)} & \hspace{1cm} & \forall k = \{1,..,K\}, \forall (f,c) \in E 
\end{eqnarray}

\end{document}
