\documentclass[12pt]{article}

\usepackage{amsmath}
\usepackage{amsfonts}
\usepackage[brazil]{babel}
\usepackage[latin1]{inputenc}
\usepackage[T1]{fontenc}
\usepackage{geometry}
\geometry{letterpaper,left=2.5cm,right=2.5cm,top=2.0cm,bottom=1.0cm}
\usepackage{graphicx}
\sloppy

\begin{document}

\begin{table}[!htb]
\begin{tabular}{rl}
\vspace{1mm}
$\lambda_r = $ &
$\left\{
\begin{array} {l}
1 - \textrm{ A rota } r \textrm{ sera usada na solu\c{c}\~ao} \\
0 - \textrm{ Caso contrario}
\end{array} \right.
$
\\
\vspace{1mm}
$\tau_i = $ &
$\left\{
\begin{array} {l}
1 - \textrm{ A mercadoria } p_i \textrm{ troca de ve\'iculo no Cross-Docking }\\
0 - \textrm{ Caso contrario }
\end{array} \right.
$
\end{tabular}
\end{table}

\[
\min \hspace{1cm} \sum\limits_{r \in R} {c_r \lambda_r} + \sum\limits_{p_i \in P} {c_i \tau_i}
\]

\begin{eqnarray}
\label{eqA1}
  \sum\limits_{r \in R} {\lambda^k_r} = K \\
\label{eqA2}
  \sum\limits_{r \in R} {a_{ir} \lambda_r} = 1 & \hspace{1cm} & \forall i \in V \\
\label{eqA3}
  \tau_i - \sum\limits_{r \in R} {b_{ir} \lambda_r} \ge 0 & \hspace{1cm} & \forall p_i \in P 
\end{eqnarray}

\vspace{3cm}

Ao resolver a relaxa\c{c}\~ao linear do problema acima e associar variaveis duais as restricoes do primal, obtem-se o modelo dual:
\begin{table}[!htb]
\begin{center}
\begin{tabular}{lll}
$(1) \rightarrow \alpha \in \mathbb{R}$ & $(2) \rightarrow \theta \in \mathbb{R}^{|V|}$ & $(3) \rightarrow \chi \in \mathbb{R}_+^{|P|}$ \\
\end{tabular}
\end{center}
\end{table}

\[
\max \hspace{1cm} K \alpha + \sum\limits_{i \in V}{\theta_i}
\]

\begin{small}
\begin{eqnarray}
\label{eqC01}
  \alpha + \sum\limits_{i \in V}{a_{ir} \theta_i} - \sum\limits_{p_i \in P}{b_{ir} \chi_i} \le c_r & \hspace{5mm} & \forall r \in R \\
\label{eqC02}
  \chi_i \le c_i & \hspace{5mm} & \forall p_i \in P 
\end{eqnarray}
\end{small}

\end{document}
