\documentclass[12pt]{article}
\usepackage{amsfonts}
\usepackage[brazil]{babel}
\usepackage[latin1]{inputenc}
\usepackage{geometry}
\geometry{letterpaper,left=2.5cm,right=2.5cm,top=2.5cm,bottom=1.5cm}
\usepackage{graphicx}
\sloppy

\begin{document}

\begin{center}
\textbf{VRP Backhals Modificado} \\
\textbf{Modelo de Fluxo em Redes} \\
\end{center}
\vspace{0.4cm}
\noindent Considere uma rede de distribuicao composta por um conjunto de fornecedores $F=\{1,2,..,N\}$ outro de consumidores $C=\{1',2',..,N'\}$, onde $|F|=|C|$. E dado tambem um conjunto $E$ de encomendas a serem distribuidas na rede, definido por uma funcao bijetiva $E:F \times C$ que associa cada elemento $f \in F$ a um unico $c \in C$. Cada encomenda $(f,c) \in E$ esta associada a um peso $P_{(f,c)}$.

\vspace{0.2cm}
Para a distribuicao das mercadorias tem-se disponivel um conjunto de veiculos $K$ todos com capacidade $P$. O problema de otimizacao consiste em definir o esquema de alocacao e roteamento dos veiculos para realizar com o minimo custo a distribuicao das mercadorias, sendo que todas devem ser coletadas nos fornecedores antes de serem entregues aos respectivos consumidores. O Fornecedores $F$ sao os vertices \textit{linehal} e os Consumidores $C$ são os \textit{backhals} do problema VRP-Backhals.

\vspace{0.2cm}
Utiliza-se um grafo $G=(V,A)$, onde $V = F \cup C \cup \{0,0'\}$, sendo $0$ o deposito e $0'$ uma copia artificial do deposito. O conjunto $A=\{(i,j), \, i,j \in V\}$ e o custo para atravessar uma aresta $(i,j) \in A$ e dado por $c_{ij}$. Para a formulacao de fluxos define-se um conjunto de commodities $Q$, cuja cardinalidade e $K$, sendo que a oferta de cada commodity $q \in Q$ estara no intervalo de $[0,|F \cup C|+1]$ unidades, disponiveis no deposito. Todo vertice $i \in F \cup C$ apresenta demanda unitaria de uma commodity que podera ser satisfeita por qualquer $q \in Q$. Alem disto, o deposito artificial $0'$ apresenta demanda de 1 unidade de toda commodity $q \in Q$. Sao consideradas as variaveis de decisao:

\vspace{0.8cm}
\begin{table}[!htb]
\begin{tabular}{rl}
$f^q_{ij} = $ & Fluxo no arco $(i,j)$ correspondente a commodity $q$ \\
$x^q_{ij} = $ &
$\left\{
\begin{array} {l}
1 - \textrm{ O arco } (i,j) \textrm{ e ativado no fluxo da commodity} q\\
0 - \textrm{ Caso contrario}
\end{array} \right.
$
\\
$y^q_i = $ &
$\left\{
\begin{array} {l}
1 - \textrm{ O vertice } i \textrm{ e ativado no fluxo da commodity } q \\
0 - \textrm{ Caso contrario}
\end{array} \right.
$
\end{tabular}
\end{table}

\newpage
\begin{center}
 $Min \,\, \sum\limits_{(i,j) \in A} {\sum \limits_{q \in Q} {c_{ij} x^{q}_{ij}}}$
\end{center}

\begin{eqnarray}
\label{eq01}
  \sum\limits_{f \in F} {f^q_{0f}} = \sum\limits_{f \in F} {y^q_f} + \sum\limits_{c \in C} {y^q_c} + 1 & \hspace{1cm} & \forall q \in Q \\
\label{eq02}
  \sum\limits_{j \in \Gamma^+_i} {f^q_{ij}} - \sum\limits_{h \in \Gamma^-_i}{f^q_{hi}} = -y^q_i & \hspace{1cm} & \forall i \in V - \{0,0'\}, \forall q \in Q \\
\label{eq03}
  \sum\limits_{q \in Q} {y^q_i} = 1 & \hspace{1cm} & \forall i \in V - \{0,0'\} \\
\label{eq04}
  y^q_f - y^q_c = 0 & \hspace{1cm} & \forall q \in Q, \forall (f,c) \in E \\  
\label{eq05}
  f^q_{ij} \le |V| x^q_{ij} & \hspace{1cm} & \forall (i,j) \in A, \forall q \in Q \\
\label{eq06}
  \sum\limits_{j \in \Gamma^+_i}{x^q_{ij}} + \sum\limits_{h \in \Gamma^-_i}{x^q_{hi}} = 2y^q_i & \hspace{1cm} & \forall i \in V - \{0,0'\}, \forall q \in Q \\
\label{eq07}
  \sum\limits_{f \in F} {y^{q}_f P_{(f,c)}} \le P & \hspace{1cm} & \forall q \in Q 
\end{eqnarray}

\end{document}
